\documentclass{naughieLuatex}
\usepackage{naughieCommands}
\title{Notes of Categories for the Working Mathematicians}
\author{Naughie}
\date{}
\newcommand\cat\mathscr
\renewcommand\defterm[1]{ \index{#1}\textbf{\emph{#1}} }
\newcommand\opair[2]{\langle #1, #2 \rangle}
\newcommand\ext\tilde
\newcommand\genby[1]{\langle #1 \rangle}
\newcommand\fmly\genby
\newcommand\thatis{\emph{i. e.}}
\newcommand\etcetra{\emph{etc.}}
\newcommand\onept\ast
\newcommand\EMPH\emph
\newcommand\catb\mathbf
\newcommand\POW{\frk P}
\newcommand\op{\mathrm{op}}
\newcommand\nattr{\stackrel{.}{\to}}
\newcommand\pushoutc{{\cdot \leftarrow \cdot \rightarrow \cdot}}
\newcommand\dLim\varinjlim
\newcommand\iLim\varprojlim
\DeclareMathOperator\colim{colim}
\newcommand\Vect{\catb{Vect}}
\newcommand\Set{\catb{Set}}
\newcommand\Grp{\catb{Grp}}
\newcommand\Grph{\catb{Grph}}
\newcommand\Cat{\catb{Cat}}
\newcommand\Dom{\catb{Dom}}
\newcommand\Fld{\catb{Fld}}
\newcommand\Met{\catb{Met}}
\newcommand\CMet{\catb{CMet}}
\newcommand\Ens{\catb{Ens}}
\newcommand\Bilin{\catb{Bilin}}
\newcommand\Mod{\catb{Mod}}
\newcommand\rMod[1][R]{\Mod_{#1}}
\newcommand\lMod[1][R]{\vphantom{\Mod}_{#1} \Mod}
\newcommand\Tops{\catb{Top}}
\newcommand\Ab{\catb{Ab}}
\newcommand\Rng{\catb{Rng}}
\newcommand\CRng{\catb{CRng}}
\newcommand\Matr{\catb{Matr}}
\begin{document}
\thispagestyle{onContentsPage}\pagestyle{onContentsPage}\pagenumbering{roman}
\maketitle\tableofcontents
\newpage\section*{Preface}\addcontentsline{toc}{section}{Preface}
これは,\href{https://twitter.com/naughiez}{@naughiez}による,\href{https://www.springer.com/us/book/9780387984032}{MacLane's Categories for the Working Mathematicians}の個人的なメモです.
\pagestyle{onDocumentsPage}\pagenumbering{arabic}

\setcounter{section}{2}

\section{Universals and Limits}
\subsection{Universal arrows}

\begin{define}
  $S \colon \cat D \to \cat C$ をfunctorとし,$c \in \cat C$ をobjectとする.$\opair{r \in \cat D}{u \colon c \to S r}$ が\defterm{universal arrow from $c$ to $S$}であるとは,任意の $\opair{d \in \cat D}{f \colon c \to S d}$ に対し,唯一つの射 $\ext f \colon r \to d$ が存在して,$f = S \ext f \circ u$ とできることをいう.
  \begin{comdia}
    r \arrow[dd, dashrightarrow, "\uniq \ext f"] & & & c \arrow[rdd, "f"] & \\
    \\
    d & & S r \arrow[ruu, leftarrow, "u"] \arrow[rr, dashrightarrow, "S \ext f"'] &   & S d
  \end{comdia}
\end{define}

\begin{align*}
  & c \xrightarrow u S r \colon \text{universal arrow} \\
  \Longleftrightarrow \quad & c \xrightarrow u S r \colon \text{initial object in} \ (c \downarrow S)
\end{align*}
である.特に,universal arrowは\EMPH{unique up to isomorphism} (if exists).

以下,特に断らない限り,$U \colon \cat D \to \cat C$ は forgetful functorとする.

\begin{exs}
\item $k$ をfield,$X$ をsetとする.$X \inc U \Span_k X$ はuniversal arrow from $X$ to $U \colon \Vect_k \to \Set$.
\item $G$ をgraphとすると,$P \colon G \to U \cat C_G$ はuniversal arrow from $G$ to $U \colon \Grph \to \Cat$.
\item $X$ をset,$\genby X$ を $X$ から生成されたfree groupとする.$X \inc U \genby X$ はuniversal arrow from $X$ to $U \colon \Grp \to \Set$.
\item $\Dom_m$ を次のように定める:
  \begin{itemize}
    \item obj.: objects of $\Dom$, \thatis, integral domains;
    \item arr.: monomorphic ring homomorphisms between integral domains.
  \end{itemize}
  Fields間のring homomorphismsはすべてmonomorphismsであることに注意して(実際,ring homomorphismのkernelは(two sided) idealであるから,自明なものしかない ),$U \colon \Fld \to \Dom_m$ が定まる.$D \in \Dom_m$ に対して $\Frac D$ をそのfield of fractionsとすれば,$D \inc U \Frac D$ はuniversal arrow from $D$ to $U$.
  \begin{rem}
    $\Dom_m$ を $\Dom$ に置き換えてはいけない!実際,$F$ をfieldとし,ring homomorphism $f \colon D \to F$ を $\ext f \colon \Frac D \to F$ に拡張するには,
    \[
      \ext f \left( \fr 1r \right) = \fr 1{f (r)}
    \]
    でなければならない.そのためには,
    \[
      r \in D \setminus \{0\} \Longrightarrow f (r) \in F \setminus \{0\}
    \]
    \thatis, $f$ はmonomorphismでなければならない.
  \end{rem}
\item $\Met$ を次のように定める:
  \begin{itemize}
    \item obj.: metric spaces;
    \item arr.: maps preserving metric.
  \end{itemize}
  $\CMet \subset \Met$ を,full subcategory whose objects are completeとする.$X$ をmetric space,$\bar X$ をそのcompletionとすると,$X \to \bar X$ はuniversal arrow from $X$ to $U \colon \CMet \to \Met$.
\end{exs}

\begin{define}
  $H \colon \cat D \to \Set$ をfunctorとする.$\opair{r \in \cat D}{e \in H r}$ が\defterm{universal element of $H$}とは,任意の $\opair{d \in \cat D}{x \in H d}$ に対して,唯一つの射 $f \colon r \to d$ が存在して,$(H f) e = x$ とできることをいう.
\end{define}

\begin{rem}
  特殊な状況では,universal arrowとuniversal elementは同じものである.
  \begin{itemize}
    \item $H \colon \cat D \to \Set$ がfunctorで,$\opair r e$がuniversal elementであるとは,$e \in H r$ を射 $\onept \xrightarrow e H r$ in $\Ens$ と見たときに,$\opair r e$ がuniversal arrow from $\onept$ to $H$ であること.
      \begin{comdia}
        r \arrow[dd, dashrightarrow, "\uniq f"] & & & \onept \arrow[rdd, "x"] & \\
        \\
        d & & H r \arrow[ruu, leftarrow, "e"] \arrow[rr, dashrightarrow, "H f"'] & & H d
      \end{comdia}
    \item 逆に,$\cat C$ をsmall categoryとし,$S \colon \cat D \to \cat C$ をfunctor,$c \in \cat C$ とすると,$\opair r u$ がuniversal arrow from $c$ to $S$ であるとは,これが $H = \cat C (c, S \pholder)$ のuniversal elementであること.
  \end{itemize}
\end{rem}

\begin{exs}
\item $S$ をset,$E \subset S \times S$ を $S$ のequivalence relation,$\pi \colon S \surj \quot S E$ とする.$\opair{\quot S E}{\pi}$ はuniversal element of $H \colon \Set \to \Set$, where
  \begin{itemize}
    \item obj.: $H X \defeq \{ f \colon S \to X \deli s E s' \Longrightarrow f s = f s' \}$;
    \item arr.: $H g \colon H X \ni f \mapsto g \circ f \in H Y$ ($g \colon X \to Y$).
  \end{itemize}
\item $G$ をgroup,$N \triangleleft G$,$\pi \colon G \surj \quot G N$ とする.$\opair{\quot G N}{\pi}$ はuniversal element of $H \colon \Grp \to \Set$, where
  \begin{itemize}
    \item obj.: $H G' \defeq \{ f \colon G \to G' \deli \text{group hom. s. t.} \ \ker f \subset N \}$;
    \item arr.: $H g \colon H G' \ni f \mapsto g \circ f \in H G''$ ($g \colon G' \to G''$).
  \end{itemize}
\item $V_1, V_2$ をvector spaces / $k$,$H \colon \Vect_k \to \Set$ を次で定まるfunctorとする:
  \begin{itemize}
    \item obj.: $H W \defeq \Bilin (V_1, V_2; W) \defeq \{ f \colon V_1 \times V_2 \to W \deli \text{bilinear} \}$;
    \item arr.: $H g \colon H W \ni f \mapsto g \circ f \in H W'$ ($g \colon W \to W'$).
  \end{itemize}
  $\opair{V_1 \otimes_k V_2}{\otimes}$ は,universal element of $H$. ($\otimes \colon V_1 \times V_2 \ni (x, y) \mapsto x \otimes y \in V_1 \otimes_k V_2$.)

  $\Vect_k$ ではなく,$\rMod$ でもよい.
\end{exs}

\begin{define}
  $S \colon \cat D \to \cat C$ をfunctorとし,$c \in \cat C$ とする.$\opair{r \in \cat D}{v \colon S r \to c}$ が\defterm{universal arrow from $S$ to $c$}であるとは,これが $(S \downarrow c)$ のterminal objectであることをいう.
  \begin{comdia}
    d \arrow[dd, dashrightarrow, "\uniq \ext f"] & & S d \arrow[rr, dashrightarrow, "S \ext f"] \arrow[rdd, "f"'] & & S r \\
    \\
    r & &     & c \arrow[ruu, leftarrow, "v"'] &
  \end{comdia}
\end{define}

\begin{exs}
\item $\cat C = \Set, \Grp, \Cat, \Tops, \Vect_k$, \etcetra\ (categories where the direct product $\times$ of two objects are defined) とする.

  $a, b \in \cat C$ を任意に取り,$p \colon a \times b \to a$, $q \colon a \times b \to b$ をcanonical projectionsとする.$\opair p q$ はuniversal arrow to $\opair a b$ from $\Delta \colon \cat C \to \cat C \times \cat C$, where
  \begin{itemize}
    \item obj.: $\Delta c \defeq \opair c c$;
    \item arr.: $\Delta f \defeq \opair f f$.
  \end{itemize}
  この $\Delta$ を\defterm{diagonal functor}という.
  \begin{comdia}
    \opair{c_1}{c_2} \arrow[rdd, "\opair{f_1}{f_2}"'] \arrow[rr, dashrightarrow, "\uniq \opair{f_1 \times f_2}{f_1 \times f_2}"] & & \opair{a \times b}{a \times b} \\
    \\
    & \opair a b \arrow[ruu, leftarrow, "\opair{p}{q}"'] &
  \end{comdia}
\end{exs}

\subsubsection*{Exercises}

\begin{enumerate}[label=(\arabic*)]
  \setcounter{enumi}{1}%
  \item The universal element of $\POW \colon \Set^\op \to \Set$ (power set) is $\opair{\{ 0, 1 \}}{1 \in \{ 0, 1 \}}$
  \item $G \in \Grp \ (\text{or}\ \in \Ab)$,$X \in \Set$ とする.The universal arrow from $G, G, X, X$, respectively, to the following forgetful functors are:
    \begin{itemize}
      \item $U \colon \Ab \to \Grp \Longrightarrow \opair{\quot{G}{[G, G]}}{\pi \colon G \surj \quot{G}{[G, G]}}$ (commutator group and abelianization),
      \item $U \colon \Rng \to \Ab \Longrightarrow \opair{R [G]}{\iota \colon G \inc R [G]}$ (group ring),
      \item $U \colon \Tops \to \Set \Longrightarrow \opair{(X, 2^X)}{\id_X \colon X \to X}$ (discrete topology),
      \item $U \colon \Set_\onept \to \Set \Longrightarrow \opair{X \amalg \{ X \}}{\iota \colon X \inc X \amalg \{ X \}}$ (one-point compactification).
    \end{itemize}
\end{enumerate}

\subsection{The Yoneda Lemma}

\begin{prop}<prop:universal-arrow-and-natural-isomorphism>
  $S \colon \cat D \to \cat C$ をfunctorとし,$c \in \cat C$ とする.$\opair r u$ がuniversal arrow from $c$ to $S$ ならば,
  \begin{align*}
    \cat D (r, d) \cong \cat C (c, S d) \quad \text{naturally in} \ d \ \text{via} \ \ext f \mapsto S \ext f \circ u.
  \end{align*}

  逆に,$\cat D (r, d) \cong \cat C (c, S d)$ naturally in $d$ ならば,唯一つの $u \colon c \to S r$ が存在して,$\opair r u$ はuniversal arrow from $c$ to $S$ である.
\end{prop}

\begin{proof}
  ($\Longrightarrow$)
  $\opair r u$ をuniversal arrow from $c$ to $S$ とする.このとき,(by definition of universal arrows)$\cat D (r, d) \cong \cat C (c, S d)$ である.これが natural in $d$ であることを見るために,$g \colon d \to d'$ とすると,$S (g \colon \ext f) \circ u = S g \circ (S \ext f \circ u)$ となる,\thatis, naturality in $d$ を示せた.
  \begin{comdia}
    & r \arrow[rdd, dashrightarrow, "\ext f'"] & & & & c \arrow[dd, "f"] \arrow[rdd, "f'"] & \\
    \\
    d \arrow[ruu, dashleftarrow, "\ext f"] \arrow[rr, "g"'] & & d' & & S r \arrow[ruu, leftarrow, "u"] \arrow[r, dashrightarrow, "S \ext f"'] \arrow[rr, bend right=50, dashrightarrow, "S \ext f'"'] & S d \arrow[r, "S g"'] & S d'
  \end{comdia}

  \begin{comdia}
    \cat D (r, d) \arrow[rr, "\varphi_d"] \arrow[dd, "{\cat D (r, g)}"'] & & \cat C (c, S d) \arrow[dd, "{\cat C (c, S g)}"] & & \ext f \arrow[rrr, mapsto] \arrow[dd, mapsto] & & & S \ext f \circ u \arrow[dd, mapsto] \\
    \\
    \cat D (r, d') \arrow[rr, "\varphi_{d'}"'] & & \cat C (c, S d') & & g \circ \ext f \arrow[rr, mapsto] & & S (g \circ \ext f) \circ u \arrow[r, equal] & S g \circ (S \ext f \circ u)
  \end{comdia}

  ($\Longleftarrow$)
  $\cat D (r, d) \xiso{\varphi_d} \cat C (c, S d)$ naturally in $d$ とする.このとき $\varphi_d$ の自然性より,$u \defeq \varphi_r 1_r$ とおけば,任意の $\opair{d \in \cat D}{f \colon c \to S d}$ に対して唯一つの $\ext f \colon r \to d$ が存在して,$f = S \ext f \circ u$ となる.実際,$\ext f \defeq S \varphi_d^{-1} f$ とおけばよい.このような $\ext f$ の一意性は,$\varphi_d$ がbijectiveであることから従う.
  \begin{comdia}
    \cat D (r, r) \arrow[rr, "\varphi_r"] \arrow[dd, "{\cat D (r, \ext f)}"'] & & \cat C (c, S r) \arrow[dd, "{\cat C (c, S \ext f)}"] & & 1_r \arrow[rrr, mapsto] \arrow[dd, mapsto] & & & u \defeq \varphi_r 1_r \arrow[dd, mapsto] \\
    \\
    \cat D (r, d) \arrow[rr, "\varphi_{d}"'] & & \cat C (c, S d) & & \ext f \arrow[rr, mapsto] & & \varphi_d \ext f \arrow[r, equal] & S \ext f \circ u
  \end{comdia}

  \begin{comdia}
    r \arrow[dd, dashrightarrow, "\uniq \varphi_d^{-1} f"] & & & c \arrow[rdd, "f"] & \\
    \\
    d & & S r \arrow[ruu, leftarrow, "u"] \arrow[rr, dashrightarrow, "S \varphi_d^{-1} f"'] & & S d
  \end{comdia}
\end{proof}

\begin{define}
  $\cat D$ をcategory whose hom-sets are smallとし,$K \colon \cat D \to \Set$ をfunctorとする.$\opair{r \in \cat D}{\psi \colon \cat D (r, \pholder) \iso K}$ を\defterm{representation of $K$}といい,$r$ を\defterm{representing object}という.A representationが存在するとき,$K$ はrepresentableであるという.
\end{define}

\tref{prop:universal-arrow-and-natural-isomorphism} より,$\cat C (c, S \pholder)$ は $\opair{r}{\varphi}$ によってrepresentsされ,従ってrepresentableである.

\begin{prop}<prop:representation-and-universal-arrow>
  $\cat D$ をcategory whose hom-sets are smallとし,$K \colon \cat D \to \Set$ をfunctorとする.もし $\opair{r \in d}{u \colon \onept \to K r}$ がuniversal arrow from $\onept$ to $K$ ならば,
  \[
    \psi_d \colon \cat D (r, d) \to K d, \quad \ext f \mapsto K (\ext f) (u \onept)
  \]
  によって定まる $\psi$ はrepresentation of $K$ である.

  逆に,$K$ の各representationは,唯一つのuniversal arrow from $\onept$ \to $K$ からこのようにして得られる.
\end{prop}

\begin{proof}
  ($\Longrightarrow$)
  $\opair r u$ をuniversal arrow from $\onept$ to $K$ とすると,\tref{prop:universal-arrow-and-natural-isomorphism} より,
  \[
    \psi_d \colon \cat D (r, d) \xiso{\varphi_d} \Set (\onept, K d) \iso K d \quad \text{naturally in}\ d
  \]
  なる自然変換 $\psi \colon \cat D (r, \pholder) \iso K$ が存在する.この対応は,
  \[
    \ext f \mapsto K \ext f \circ u \mapsto (K \ext f \circ u) (\onept) = K (\ext f) (u \onept)
  \]
  で与えられる.

  ($\Longleftarrow$)
  逆に,$\opair{r}{\psi}$ をrepresentation of $K$ とする.
  \[
    \varphi_d \colon \cat D (r, d) \xiso{\psi_d} K d \iso \Set (\onept, K d) \quad \text{naturally in}\ d
  \]
  によってnatural isomorphism $\varphi \colon \cat D (r, \pholder) \iso \Set (\onept, K \pholder)$ を定義すれば,再び \tref{prop:universal-arrow-and-natural-isomorphism} より,唯一つの $u \colon \onept \to K r$ が存在して,$\opair{r}{u}$ はuniversal arrow from $\onept$ to $K$ となる.このとき $K$ は,
  \[
    \psi_d \colon \cat D (r, d) \xiso{\varphi_d} \Set (\onept, K d) \iso K d
  \]
  によってrepresentsされる.
\end{proof}

\begin{lem}[\textgt{米田の補題}(\emph{Yoneda Lemma})]<lem:yoneda>
  $\cat D$ をcategory whose hom-sets are smallとし,$K \colon \cat D \to \Set$ をfunctorとする.このとき,
  \[
    \exists y \colon \Nat (\cat D (r, \pholder), K) \iso K r, \quad \alpha \mapsto \alpha_r 1_r.
  \]
\end{lem}

\begin{proof}
  $y \colon \Nat (\cat D (r, \pholder), K) \ni \alpha \mapsto \alpha_r 1_r \in K r$ の逆射を求めればよい.今 $e \in K r$ が任意に与えられたとする.The natural transformation $\alpha \colon \cat D (r, \pholder) \nattr K$ を次のように定める:
  \[
    \alpha_d \colon \cat D (r, d) \to K d, \quad \ext f \mapsto K (\ext f) (e).
  \]
  このとき $\alpha_r 1_r = K (1_r) (e) = e$ であり,さらに
  \begin{comdia}
    d \arrow[dd, "g"] & & \cat D (r, d) \arrow[rr, "\alpha_d"] \arrow[dd, "{\cat D (r, g)}"'] & & K d \arrow[dd, "K g"] & & \ext f \arrow[dd, mapsto] \arrow[rrr, mapsto] & & & K (\ext f) (e) \arrow[dd, mapsto] \\
    \\
    d' & & \cat D (r, d') \arrow[rr, "\alpha_{d'}"'] & & K d' & & g \circ \ext f \arrow[rr, mapsto] & & K (g \circ \ext f) (e) \arrow[r, equal] & K (g) (K (\ext f) (e))
  \end{comdia}
  となる.よって,$\alpha \in \Nat (\cat D(r, \pholder), K)$ で,$y \alpha = e$,\thatis, $y$ is bijective. ($\alpha$ の一意性は \tref{prop:representation-and-universal-arrow} から分かる.)
\end{proof}

\begin{cor}<cor:corollary-to-yoneda-lemma>
  $\cat D$ をcategory whose hom-sets are smallとし,$r, s \in \cat D$ とする.このとき,任意のnatural transformation $\alpha \colon \cat D (r, \pholder) \nattr \cat D (s, \pholder)$ に対して,唯一つの射 $h \colon s \to r$ が存在して,$\alpha = \cat D (h, \pholder)$ となる,\thatis,
  \[
    \alpha_d = \cat D (h, d) \colon \cat D (r, d) \to \cat D (s, d), \quad f \mapsto f \circ h.
  \]
\end{cor}

\begin{proof}
  \tref{lem:yoneda} とその証明をfunctor $K = \cat D (s, \pholder)$ に適用すれば,$\alpha$ に対して唯一つの $h \in \cat D (s, r)$ が存在して,
  \[
    \alpha_d \colon \cat D (r, d) \to \cat D (s, d), \quad f \mapsto \cat D (s, f) h = f \circ h
  \]
  とできる.
\end{proof}

\tref{lem:yoneda} の全単射 $y \colon \Nat (\cat D (r, \pholder), K) \iso K r$ は,実はnatural transformationである.それを述べるために,まず $\Nat (\cat D (r, \pholder), K)$ と $K r$ が,$K$ と $r$ に関するfunctorsと見なせることを確認する.
\begin{itemize}
  \item $K r$: evaluation functor $E \colon \Set^{\cat D} \times \cat D \to \Set$, where
    \begin{itemize}
      \item obj.: $E \opair K r = K r$;
      \item arr.: $E \opair{\alpha \colon K \nattr K'}{f \colon r \to r'} = K' f \circ \alpha_r = \alpha_{r'} \circ K f \colon K r \to K' r'$
    \end{itemize}
  \item $\Nat (\cat D (r, \pholder), K)$: functor $N \colon \Set^{\cat D} \times \cat D \to \Set$, where
    \begin{itemize}
      \item obj.: $N \opair K r \defeq \Nat (\cat D (r, \pholder), K)$;
      \item arr.: $K \opair{\alpha \colon K \nattr K'}{f \colon r \to r'} \colon \Nat (\cat D (r, \pholder), K) \to \Nat (\cat D (r', \pholder), K'), \ \beta \mapsto \alpha \beta \cat D (f, \pholder)$.
    \end{itemize}
\end{itemize}

\begin{comdia}
  \cat D \arrow[rrr, bend left=50, "{\cat D (r', \pholder)}", ""'{name=A}] \arrow[rrr, bend right=50, "{\cat D (r, \pholder)}"', ""{name=B}] & & & \Set \arrow[Rightarrow, from=A, to=B, "{\cat D (f, \pholder)}"]  & \cat D \arrow[rrr, bend left=50, "{\cat D (r, \pholder)}", ""'{name=C}] \arrow[rrr, bend right=50, "{K}"', ""{name=D}] & & & \Set \arrow[Rightarrow, from=C, to=D, "\beta"] & \cat D \arrow[rrr, bend left=50, "{K}", ""'{name=E}] \arrow[rrr, bend right=50, "{K'}"', ""{name=F}] & & & \Set \arrow[Rightarrow, from=E, to=F, "\alpha"]
\end{comdia}

\begin{lem}
  \tref{lem:yoneda} における全単射
  \[
    y = y_{\opair K r} \colon \Nat (\cat D (r, \pholder), K) \iso K r
  \]
  は,natural isomorphism $N \xiso{y_{\opair K r}} E$ を定める.すなわち,$y$ はnatural in $K$ and $r$ である.
\end{lem}

\begin{proof}
  $E$ と $N$ の構成から,任意の $\opair K r, \opair{K'}{r'} \in \Set^{\cat D} \times \cat D$ と $\opair \alpha f \colon \opair K r \to \opair{K'}{r'}$ に対して,$\cat D (f, r') 1_{r'} = f = \cat D (r, f) 1_r$ に注意すれば,以下の可換図式から従う:
  \begin{comdia}
    \Nat (\cat D (r, \pholder), K) \arrow[rr, "{y_{\opair K r}}"] \arrow[dd, "{N \opair \alpha f}"'] & & K r \arrow[dd, "{E \opair \alpha f}"] & & \beta \arrow[rrr, mapsto] \arrow[dd, mapsto] & & & \beta_r 1_r \arrow[dd, mapsto] \\
    \\
    \Nat (\cat D (r', \pholder), K') \arrow[rr, "{y_{\opair{K'}{r'}}}"'] & & K' r' & & \alpha \beta \cat D (f, \pholder) \arrow[rr, mapsto] & & \alpha_{r'} \beta_{r'} \cat D (f, r') (1_{r'}) \arrow[r, equal] & (\alpha_{r'} \circ K f ) (\beta_r 1_r)
  \end{comdia}

  \begin{comdia}
    \cat D (r, r) \arrow[dd, "{\cat D (r, f)}"'] \arrow[rr, "\beta_r"] & & K r \arrow[dd, "K f"] & & 1_r \arrow[rrr, mapsto] \arrow[dd, mapsto] & & & \beta_r 1_r \arrow[dd, mapsto] \\
    \\
    \cat D (r, r') \arrow[rr, "\beta_{r'}"'] & & K r' & & f \arrow[rr, mapsto] & & \beta_{r'} f \arrow[r, equal] & K (f) (\beta_r 1_r)
  \end{comdia}
\end{proof}

The functor $Y \colon \cat D^\op \to \Set^{\cat D}$, defined as
\begin{itemize}
  \item obj.: $Y r \defeq \cat D (r, \pholder)$,
  \item arr.: $Y f \defeq \cat D (f, \pholder) \colon \cat D (r, \pholder) \nattr \cat D (s, \pholder)$ ($f \colon s \to r$),
\end{itemize}
を\defterm{Yoneda functor}という.これはfull and faithfulである.実際,full であることも,faithfulであることも,\tref{cor:corollary-to-yoneda-lemma} から従う.

Yoneda functorの双対 $Y' \colon \cat D \to \Set^{\cat D^\op}$ は,次で定まる:
\begin{itemize}
  \item obj.: $Y' r \defeq \cat D (\pholder, r)$;
  \item arr.: $Y' f \defeq \cat D (\pholder, f) \colon \cat D (\pholder, s) \nattr \cat D (\pholder, r)$ ($f \colon s \to r$).
\end{itemize}
これはfaithfulである.実際,$f, f' \colon s \rightrightarrows r$ が $Y' f = Y' f'$ とすると,
\[
  f = (Y' f)_s 1_s = (Y' f')_s 1_s = f'.
\]

逆に,これらのfunctor $Y, Y'$ が定義できるのなら,$\cat D$ はcategory whose hom-sets are smallでなければならない.なぜなら,このとき任意の $r, s \in \cat D$ に対して,$\cat D (r, s) = (Y r) s = (Y' s)r \in \Set$ はsmallだからである.
より大きい $\cat D$ に対しても,$\Set$ を $\Ens$ に置き換えたものが同様に成立する.

\subsection{Coproducts and Colimits}\label{subsec:coproducts-and-colimits}

\subsubsection*{Coproducts}

\begin{define}
  $\cat C$ をcategoryとし,$a, b \in \cat C$ とする.$\opair{c \in \cat C}{\opair i j \colon \opair a b \to \opair c c }$ が\defterm{coproduct of $a$ and $b$}であるとは,$\opair{c}{i, j}$ がuniversal arrow from $\opair a b$ to the diagonal functor $\Delta$ であることをいう.このとき,
  \[
    c = a \amalg b = a \oplus b
  \]
  などと書く.
  $i, j$ を\defterm{injections}という.

  \begin{comdia}
    a \arrow[rr, "i"] \arrow[rrdd, "f"'] & & a \amalg b \arrow[rr, leftarrow, "j"] \arrow[dd, dashrightarrow, "\uniq h"] & & b \\
    \\
    & & d \arrow[rruu, leftarrow, "g"']
  \end{comdia}
\end{define}

$a, b$ のcoproductについて,以下の全単射がある:
\[
  \cat C (a, d) \times \cat C (b, d) \iso \cat C (a \amalg b, d), \quad \opair f g \mapsto h, \quad \text{naturally in}\ d,
\]
with inverse $h \mapsto \opair{h i}{h j}$.

\begin{rem}
  $\cat C$ をcategory every two of whose objects have the coproductとすると,bifunctor $\amalg \colon \cat C \times \cat C \to \cat C$ を定める:
  \begin{itemize}
    \item obj.: $\opair a b \mapsto a \amalg b$;
    \item arr.: $\opair h k \mapsto h \amalg k$, defined by the following diagram:
  \end{itemize}
  \begin{comdia}
    a \arrow[rr, "i"] \arrow[dd, "h"'] \arrow[rrdd, "i' h"'] & & a \amalg b \arrow[rr, leftarrow, "j"] \arrow[dd, dashrightarrow, "\uniq h \amalg k"] & & b \arrow[dd, "k"] \\
    \\
    a' \arrow[rr, "i'"'] & & a' \amalg b' \arrow[rruu, leftarrow, "j' k"'] \arrow[rr, leftarrow, "j'"'] & & b'
  \end{comdia}
\end{rem}

\begin{exs}
\item $\Set$: disjoint union $a \inc a \amalg b \rinc b$.
\item $\Tops$: disjoint union $a \inc a \amalg b \rinc b$.
\item $\Tops_\onept$: wedge product $\opair{a}{\onept_a} \inc \quot{a \amalg b}{\genby{\onept_a = \onept_b}} \rinc \opair{b}{\onept_b}$.
\item $\Ab, \lMod$: direct sum $a \inc a \oplus b \rinc b$.
\item $\Grp$: free group $a \inc \genby{a, b} \rinc b$.
\item $\CRng$: tensor product $a \inc a \otimes b \rinc b$.
\item $P$ (preordered set): least upper bound $a \le a \cup b \ge b$.
\end{exs}

\subsubsection*{Infinite Coproducts}
$X$ をsetとする.
$\Delta \colon \cat C \to \cat C^X$ を\defterm{diagonal functor}, where
\begin{itemize}
  \item obj.: $\Delta c \defeq \fmly{c_x = c}_{x \in X}$,
  \item arr.: $\Delta f \defeq \fmly{f_x = f}_{x \in X}$,
\end{itemize}
とする.

\begin{define}
  $\cat C$ をcategory,$X$ をsetとし,$a = \fmly{a_x}_{x \in X} \in \cat C^X$ とする.$\opair{c \in \cat C}{i_x \colon a_x \to c, x \in X}$ が\defterm{coproduct of $a$}であるとは,$\opair{c}{i_x, x \in X}$ がuniversal arrow from $a$ to $\Delta$ であることをいう.このとき,
  \begin{align*}
    c &= \coprod_{x \in X} a_x = \coprod a_x = \coprod a \\
      &= \bigoplus_{x \in X} a_x = \bigoplus a_x = \bigoplus a \\
  \end{align*}
  などと書く.
  \begin{comdia}
    a_x \arrow[rr, "i_x"] \arrow[rrdd, "f_x"'] & & \coprod_{x'} a_{x'} \arrow[dd, dashrightarrow, "\uniq \ext f"] \\
    \\
    & & d
  \end{comdia}
\end{define}
$a = \fmly{a_x} \in \cat C^X$ のcoproductについて,以下の全単射がある:
\[
  \prod_{x \in X} \cat C (a_x, d) \iso \cat C \left( \coprod_{x \in X} a_x, d \right), \quad \fmly{f_x} \mapsto \ext f, \quad \text{naturally in} \ d,
\]
with inverse $f \mapsto \fmly{f i_x}$.

\begin{rem}
  $\cat C$ をcategory every $X$-fold family of whose objects have the coproductとすると,$X$-fold functor $\coprod \colon \cat C^X \to \cat C$ を定める:
  \begin{itemize}
    \item obj.: $\fmly{a_x} \mapsto \coprod a_x$;
    \item arr.: $\fmly{h_x} \mapsto \coprod h_x$, defined by the following diagram:
  \end{itemize}
  \begin{comdia}
    a_x \arrow[rr, "i_x"] \arrow[dd, "h_x"'] \arrow[rrdd, "j_x h_x"'] & & \coprod a_{x'}\arrow[dd, dashrightarrow, "\uniq \coprod h_{x'}"] \\
    \\
    b_x \arrow[rr, "j_x"'] & & \coprod b_{x'}
  \end{comdia}
\end{rem}

\subsubsection*{Copowers}

$b = \fmly{b_x = a}_{x \in X} \in \cat C^X$ のcoproductを\defterm{copower of $a$}といい,$X \cdot a \defeq \coprod_x a$ と書く.このとき,
\[
  \cat C (a, d)^X \iso \cat C (X \cdot a, d), \quad \text{naturally in}\ d.
\]

\begin{exs}
\item $\Set$: $a = Y$ のcopowerは,$X \cdot Y = X \times Y$.
\end{exs}

\subsubsection*{Cokernels}

\begin{define}
  $\cat C$ をcategory that has the null object $z$ とし,$f \colon a \to b$ とする.$\opair{e \in \cat C}{u \colon b \to e}$ が\defterm{cokernel of $f$}であるとは,$u f = 0$ であって,任意の射 $h \colon b \to c$ with $h f = 0$ に対して,唯一つの射 $\ext h \colon e \to c$ が存在して,$h = \ext h \circ u$ とできることをいう.
  \begin{comdia}
    a \arrow[rr, "f"] & & b \arrow[rr, "u"] \arrow[rrdd, "h"'] & & e \arrow[dd, dashrightarrow, "\uniq \ext h"] \\
    \\
    & & & & c
  \end{comdia}
\end{define}

The cokernel $\opair r u$ of $f$ とは,universal element of the functor $H \colon \cat C \to \Set$, where
\[
  H c \defeq \{ h \colon b \to c \deli h f = 0 \},
\]
に他ならない.

\begin{exs}
\item $\cat C = \Grp, \Ab, \Rng, \lMod$, \etcetra, とする.$\cat C$ におけるcokernel of $f \colon a \to b$ は,通常の意味でのcokernelである.すなわち,$\opair{\coker f \defeq \quot{b}{\im f}}{\pi \colon b \surj \coker f}$ のこと.
\end{exs}

\subsubsection*{Coequalizers}

\begin{define}
  $\cat C$ をcategoryとし,$f, g \colon a \rightrightarrows b$ を $\cat C$ の射とする.$\opair{e \in \cat C}{u \colon b \to e}$ が\defterm{coequalizer of $\opair f g$}であるとは,$u f = u g$ であって,任意の射 $h \colon b \to c$ with $h f = h g$ に対して,唯一つの射 $\ext h \colon e \to c$ が存在して,$h = \ext h u$ とできることをいう.
  \begin{comdia}
    a \arrow[rr, shift left, "f"] \arrow[rr, shift right, "g"'] & & b \arrow[rr, "u"] \arrow[rrdd, "h"'] & & e \arrow[dd, dashrightarrow, "\uniq \ext h"] \\
    \\
    & & & & c
  \end{comdia}
\end{define}

\begin{rem}
  The coequalizerをuniversal arrowと考えることもできる.$\downdownarrows$ を,category $\cdot \rightrightarrows \cdot$ that has only two objectsとする.このとき,$\cat C^\downdownarrows$ は,category consisting of
  \begin{itemize}
    \item obj.: 射 $\opair f g \colon a \rightrightarrows b$;
    \item arr.: 射 $\opair f g \colon a \rightrightarrows b$ と $\opair{f'}{g'} \colon a' \rightrightarrows b'$ に対して,$\opair{h \colon a \to a'}{k \colon b \to b'}$ s.t. $k f = f' h$ and $k g = g' h$.
  \end{itemize}
  \begin{comdia}
    a \arrow[rr, shift left, "f"] \arrow[rr, shift right, "g"'] \arrow[dd, "h"'] & & b \arrow[dd, "k"] \\
    \\
    a' \arrow[rr, shift left, "f'"] \arrow[rr, shift right, "g'"'] & & b'
  \end{comdia}
  $\Delta \colon \cat C \to \cat C^\downdownarrows$ をdiagonal functor, defined as:
  \begin{itemize}
    \item obj.: $\Delta c \defeq \opair{1_c}{1_c}$,
    \item arr.: $\Delta r \defeq \opair r r \colon \opair{1_c}{1_c} \to \opair{1_{c'}}{1_{c'}}$ ($r \colon c \to c'$),
  \end{itemize}
  \begin{comdia}
    c \arrow[dd, "r"'] & & c \arrow[rr, shift left, "1_c"] \arrow[rr, shift right, "1_c"'] \arrow[dd, "r"'] & & c \arrow[dd, "r"] \\
    \\
    c' & & c' \arrow[rr, shift left, "1_{c'}"] \arrow[rr, shift right, "1_{c'}"'] & & c'
  \end{comdia}
  とする.

  このとき,射 $h \colon b \to c$ with $h f = h g$, where $f, g \colon a \rightrightarrows b$, とは,$\cat C^\downdownarrows$ の射 $\opair{h f}{h} \colon \opair f g \to \opair{1_c}{1_c}$ のことである.
  \begin{comdia}
    a \arrow[rr, shift left, "f"] \arrow[rr, shift right, "g"'] \arrow[dd, "h f"'] & & b \arrow[dd, "h"] \\
    \\
    c \arrow[rr, shift left, "1_c"] \arrow[rr, shift right, "1_c"'] & & c
  \end{comdia}

  従って,$\opair e u$ がcoequalizer of $\opair f g$ であるとは,universal arrow from $\opair f g$ to $\Delta$ に他ならない.
  \begin{comdia}
    e \arrow[dd, dashrightarrow, "\uniq \ext h"] & & & \opair f g \arrow[rdd, "\opair{h f}{h}"] \\
    \\
    c & & \opair{1_e}{1_e} \arrow[ruu, leftarrow, "\opair{u f}{u}"] \arrow[rr, dashrightarrow, "\opair{\ext h}{\ext h}"'] & & \opair{1_c}{1_c}
  \end{comdia}
\end{rem}

任意の平行射の族 $\fmly{f_x \colon a \to b}_{x \in X}$ についても同様に定義できる.すなわち,$\opair e u$ がcoequalizer of $\fmly{f_x}$ であるとは,$u f_x = u f_y$ ($\forall x, y \in X$) であって,任意の射 $h \colon b \to c$ with $h f_x = h f_y$ ($\forall x, y \in X$) に対して,唯一つの射 $\ext h \colon e \to c$ が存在して,$h = \ext h u$ とできることをいう.

\begin{exs}
\item $\cat C = \Grp, \Ab, \Rng, \lMod$, \etcetra, とする.$f, g \colon a \rightrightarrows b$ のcoequalizerは,$f - g$ のcokernel $b \surj \coker (f - g)$ である.
\item $\cat C = \Set, \Tops$, \etcetra, とする.$f, g \colon a \rightrightarrows b$ のcoequalizerは,$\opair{\quot b R}{\pi \colon b \surj \quot b R}$, where
  \[
    R \defeq \{ \opair{f x}{g x} \deli x \in a \} \subset b \times b,
  \]
  である.
\end{exs}

\subsubsection*{Pushouts}

\begin{define}
  $\cat C$ をcategoryとし,$f \colon a \to b$, $g \colon a \to c$ とする.$\opair{r \in \cat C}{u \colon b \to r, v \colon c \to r}$ がpushout of $\opair f g$ であるとは,$u f = v g$ であって,任意の $\opair{s \in \cat C}{h \colon b \to s, k \colon c \to s}$ with $h f = k g$ に対して,唯一つの射 $t \colon r \to s$ が存在して,$h = t u$ かつ $k = t v$ とできることをいう.
  \[
    r = b \amalg_a c = b \amalg_{\opair f g} c
  \]
  などと表す.
  \begin{comdia}
    & & b \arrow[rrdd, "u"] \arrow[rrrrdd, bend left=20, "h"] \\
    \\
    a \arrow[rruu, "f"] \arrow[rrdd, "g"'] & & & & r \arrow[rr, dashrightarrow, "\uniq t"] & & s \\
    \\
                                          & & c \arrow[rruu, "v"'] \arrow[rrrruu, bend right=20, "k"']
  \end{comdia}
\end{define}

\begin{rem}
  The pushout of $\opair f g$ をuniversal arrowとして捉える.$\cat C^\pushoutc$ は,category consisting of
  \begin{itemize}
    \item obj.: $\opair h k \colon \opair a a \to \opair b c$,
    \item arr.: $\opair h k \to \opair{h'}{k'}$, s.t.
      \begin{comdia}
        b \arrow[dd] \arrow[rr, leftarrow, "h"] & & a \arrow[dd] \arrow[rr, "k"] & & c \arrow[dd] \\
        \\
        b' \arrow[rr, leftarrow, "h'"'] & & a' \arrow[rr, "k'"'] & & c'
      \end{comdia}
  \end{itemize}
  である.$\Delta \colon \cat C \to \cat C^\pushoutc$ を,diagonal functor, where
  \begin{itemize}
    \item obj.: $\Delta r \defeq \opair{1_r}{1_r}$,
    \item arr.: $\Delta h \defeq \fmly{h, h, h} \colon \opair{1_r}{1_r} \to \opair{1_s}{1_s}$, ($h \colon r \to s$),
  \end{itemize}
  とする.このとき,$\opair h k \colon \opair b c \to \opair s s$ with $h f = k g$ とは,射 $\fmly{h f, h, k} \colon \opair f g \to \Delta s$ のことである.
  \begin{comdia}
    b \arrow[dd, "h"'] \arrow[rr, leftarrow, "f"] & & a \arrow[dd, "h f"] \arrow[rr, "g"] & & c \arrow[dd, "k"] \\
    \\
    s \arrow[rr, leftarrow, "1_s"'] & & s \arrow[rr, "1_s"'] & & s
  \end{comdia}
  従って,pushout $\opair{r}{u, v}$ of $\opair f g$ とは,universal arrow from $\opair f g$ to $\Delta$ のことである.
\end{rem}

\begin{exs}
\item $\cat C = \Set, \Tops, \Grp, \Ab, \Rng, \lMod$, \etcetra, とする.$\opair f g \colon \opair a a \to \opair b c$ のpushoutは,$b \amalg c$ において,$f x \in b$ と $g x \in c$ を同一視したものである.(演習問題(2)参照.)
\end{exs}

\subsubsection*{Cokernel Pair}

\begin{define}
  $\cat C$ をcategory,$f \colon a \to b$ とする.The \defterm{cokernel pair of $f$}とは,pushout of $\opair f f$ のことである,\thatis,$r \in \cat C$ と $u, v \colon b \rightrightarrows r$ with $u f = v f$ であって,任意の射 $h, k \colon b \rightrightarrows s$ with $h f = k f$ に対して,唯一つの射 $t \colon r \to s$ が存在して,$t u = h$ かつ $t v = k$ とできることをいう.
  \begin{comdia}
    a \arrow[rr, "f"] & & b \arrow[rr, shift left, "u"] \arrow[rr, shift right, "v"'] \arrow[rrdd, shift left, "h"] \arrow[rrdd, shift right, "k"'] & & r \arrow[dd, dashrightarrow, "\uniq t"] \\
    \\
    & & & & s
  \end{comdia}
\end{define}

\subsubsection*{Colimits}

今までの議論を一般化する.$\cat C, J$ をcategoriesとする.$\Delta \colon \cat C \to \cat C^J$ を,次のように定まるfunctor, called \defterm{diagonal functor}, とする:
\begin{itemize}
  \item obj.: $\Delta c \colon J \to \cat C$, constant functor, \thatis, $(\Delta c) i = c$ and $(\Delta c) u = 1_c$ ($i, j \in \cat C$, $u \colon i \to j$);
  \item arr.: $\Delta f \colon \Delta c \nattr \Delta c'$, constant transformation, \thatis, $(\Delta f)i = f \colon c \to c'$ ($f \colon c \to c'$, $i \in J$).
\end{itemize}

このとき,natural transformation $\tau \colon F \nattr \Delta c$ は,射の族 $\fmly{\tau_i \colon F_i \to c}_{i \in J}$ であって,各 $u \colon i \to j$ に対して $\tau_i = \tau_j \circ F u$ なるものである.

\begin{comdia}
  F_i \arrow[dd, "F u"'] \arrow[rrd, "\tau_i"] \\
  & & c \\
  F_j \arrow[rru, "\tau_j"']
\end{comdia}

これを,\defterm{cone from base $F$ to the vertex $c$}という.

\begin{define}
  $\cat C, J$ をcategoriesとし,$F \colon J \cat C$ をfunctorとする.A \defterm{colimit of $F$} (or, a \defterm{direct limit}, a \defterm{inductive limit} of $F$) とは,universal arrow $\opair{r \in \cat C}{\mu \colon F \nattr \Delta r}$ from $F$ to $\Delta$  のことである.
  \[
    r = \dLim F = \colim F
  \]
  などと書く.$\mu$ を\defterm{limiting cone}, or \defterm{universal cone} from $F$ という.
  \begin{comdia}
    F_i \arrow[dd, "F u"'] \arrow[rrd, "\mu_i"] \arrow[rrrrd, bend left=20, "\tau_i"] \\
    & & \dLim F \arrow[rr, dashrightarrow, "\uniq t"] & & c \\
    F_j \arrow[rru, "\mu_j"'] \arrow[rrrru, bend right=20, "\tau_j"']
  \end{comdia}
\end{define}

\begin{exs}
\item $J = \omega = \{ 0 \to 1 \to 2 \to \dotsb \}$, $\cat C = \Set$ とする.A functor $F \colon \omega \to \Set$ とは,昇鎖
  \[
    F_0 \subset F_1 \subset F_2 \subset \dotsb
  \]
  のことである.このとき,
  \[
    \dLim F = U \defeq \bigcup_n F_n
  \]
  であり,inclusions $F_i \inc U$ がlimiting coneとなる.$\cat C = \Grp, \Ab$, \etcetra, でも同様.
\end{exs}

\subsubsection*{Exercises}
\begin{enumerate}[label=(\arabic*)]
  \setcounter{enumi}{1}%
  \item $\cat C$ をcategory that has coproduct and coequalizer of every pair of two objectsとし,$\opair f g \colon \opair a a \to \opair b c$ とする.このとき,pushout of $\opair f g$ は存在する.まず,coproductの普遍性より,任意の $\opair h k \colon \opair b c \to \opair s s$ に対して,唯一つの射 $q \colon b \amalg c \to s$ が存在して,$q i = h$ かつ $q j = k$ となる.
    \begin{comdia}
      & & b \arrow[rrdd, "i"] \arrow[rrrrdd, bend left=20, "h"] \\
      \\
      a \arrow[rruu, "f"] \arrow[rrdd, "g"'] \arrow[rrrr, shift left, "i f"] \arrow[rrrr, shift right, "j g"'] & & & & b \amalg c \arrow[rr, dashrightarrow, "\uniq q"] & & s \\
      \\
      & & c \arrow[rruu, "j"'] \arrow[rrrruu, bend right=20, "k"']
    \end{comdia}
    The coequalizer of $\opair{i f}{j g} \colon a \rightrightarrows b \amalg c$ が存在し,それを $p \colon b \amalg c \to r$ とおく.
    \begin{comdia}
      a \arrow[rr, shift left, "i f"] \arrow[rr, shift right, "j g"'] & & b \amalg c \arrow[rr, "p"] \arrow[rrdd, "q"'] & & r \arrow[dd, dashrightarrow, "\uniq t"] \\
      \\
      & & & & s
    \end{comdia}
    すると,$\opair{r}{p i, p j}$ がpushout of $\opair f g$ になる.実際,$(p i) f = (p j) g$ は定義から明らかである.任意の射 $\opair h k \colon \opair b c \to \opair s s$ with $h f = k g$ に対して,$q i = h$ かつ $q j = k$ なる唯一の射 $q \colon b \amalg c \to s$ を取れば,
    \[
      q (i f) = (q i) f = h f = k g = (q j) g = q (j g)
    \]
    であるから,coequalizerの普遍性より $q = t p$ なる唯一の射 $t \colon r \to s$ が存在する.このとき,$t (p i) = q i = h$ かつ $t (p j) = q j = k$ となる.よって,$\opair{r}{p i, p j}$ はpushout of $\opair f g$ である.
  \item $\Matr_k$, category consisting of
    \begin{itemize}
      \item obj.: $n \in \Z_{\ge 0}$,
      \item arr.: $m \times n$ matrix $A \colon n \to m$,
    \end{itemize}
    を考え,$A, B$ を $m \times n$ matricesとする.$T \defeq A - B$ とおく.$A, B$ のcoequalizerは,$(m - \rank T) \times m$ matrix
    \[
      P \defeq ( 1_{m - \rank T}, 0_{\rank T} )
    \]
    である.($1_n$ と $0_n$ は,それぞれ $n \times n$ の単位行列と零行列を表す.)$A \colon n \to m$ を,linear map $\R^n \to \R^m$ と考えると分かりやすい.このとき,(linear mapsとして)$C \colon \R^m \to \R^p$ に対して,
    \[
      C A = C B \Longleftrightarrow C T = 0 \Longleftrightarrow \uniq \ext C \colon \coker T \to \R^p \ \text{s.t.} \ C = \ext C P
    \]
    が成り立つ.$\coker T = \quot{\R^m}{\im T} \iso \R^{m - \rank T}$ であるから,最初の主張が確かめられる.
  \setcounter{enumi}{4}%
  \item $X$ をsetとし,$E \subset X \times X$ をそのequivalence relationとする.$\quot X E$は,以下の図式で定まるcoequalizerである:
    \begin{comdia}
      E \arrow[rr, hookrightarrow] & & X \times X \arrow[rr, shift left, twoheadrightarrow, "\pi_1"] \arrow[rr, shift right, twoheadrightarrow, "\pi_2"'] & & X \arrow[rr, twoheadrightarrow, "\pi"] \arrow[rrdd, "f"'] & & \quot X E \arrow[dd, dashrightarrow, "\uniq \ext f"] \\
      \\
      & & & & & & Y
    \end{comdia}
    ここで,$\pi_i \colon X \times X \surj X$ は $i$-th componentへの射影($i = 1, 2$),$\pi \colon X \surj \quot X E$ は標準的全射である.
  \item $a, b \in \cat C$ のcoproductは,functor
    \[
      \cat C (a, \pholder) \times \cat C (b, \pholder) \colon \cat C \to \Set, \quad c \mapsto \cat C (a, c) \times \cat C (b, c)
    \]
    がrepresentableであるとき,かつそのときに限り存在する.実際,このfunctorがrepresentableであることは,
    \[
      \opair{r \in \cat C}{\psi \colon \cat C (r, \pholder) \iso \cat C (a, \pholder) \times \cat C (b, \pholder)}
    \]
    が存在することであるから,$r = a \amalg b$ が存在すればrepresentableである.

    逆に,このような $\opair r \psi$ が存在すれば,$r$ がcoproduct of $a$ and $b$ together with injection $\psi_r 1_r = \opair i j$ となる($h \defeq \psi^{-1} \opair f g$):
    \begin{comdia}
      \cat C (r, r) \arrow[rr, "\psi_r"] \arrow[dd, "{\cat C (r, h)}"'] & & \cat C (a, r) \times \cat C (b, r) \arrow[dd, "{\cat C (a, h) \times \cat C (b, h)}"] & & 1_r \arrow[rrr, mapsto] \arrow[dd, mapsto] & & & \opair i j \arrow[dd, mapsto] \\
      \\
      \cat C (r, d) \arrow[rr, "\psi_d"'] & & \cat C (a, d) \times \cat C (b, d) & & h \arrow[rr, mapsto] & & \opair f g \arrow[r, equal] & \opair{h i}{h j}
    \end{comdia}
    \begin{comdia}
      a \arrow[rr, "i"] \arrow[rrdd, "f"'] & & a \amalg b \arrow[rr, leftarrow, "j"] \arrow[dd, dashrightarrow, "{\uniq h}"] & & b \\
      \\
      & & d \arrow[rruu, leftarrow, "g"']
    \end{comdia}
  \item $A$ をabelian group,
    \[
      J_A \defeq \{ H \le A \deli \text{finitely generated} \}
    \]
    とおくと,$J_A$ は包含関係について前順序集合(従ってcategory)となる.$F \colon J_A \to \Ab$ を,次で定義されるfunctorとする:
    \begin{itemize}
      \item $F H \defeq H \in \Ab$ ($H \in J_A$);
      \item $F \iota_{H, H'} = \iota_{H, H'} \colon F H \inc F H'$ ($\iota_{H, H'} \colon H \inc H'$ in $J_A$).
    \end{itemize}
    このとき,$A = \dLim F$ である.実際,任意のabelian group $G$ とgroup homomorphisms $f_H \colon H \to G$ with $f_H = f_{H'} \iota_{H, H'}$ ($\forall \iota_{H, H'} \colon H \inc H'$ in $J_A$) に対して,
    \[
      \ext f x \defeq f_H x \quad \text{if} \ x \in H \subset A
    \]
    と定めれば,これが $\ext f \circ \iota_H = f_H$ ($\forall \iota_H \colon H \in J_A \inc A$) なる唯一のgroup homomorphismである.($A = \bigcup_{H \in J_A} J_A$ に注意せよ.)

\end{enumerate}

\subsection{Products and Limits}

ここでは,\S \ref{subsec:coproducts-and-colimits} の双対を述べる.

$\cat C, J$ をcategory,$F \colon J \to \cat C$ をfunctorとする.A natural transformation $\tau \colon \Delta c \nattr F$ を\defterm{cone to the base $F$ from the vertex $c$}という.

\begin{define}
  $\cat C, J$ をcategory,$F \colon J \to \cat C$ をfunctorとする.A \defterm{limit of $F$} (or, a \defterm{inverse limit}, a \defterm{projective limit} of $F$) とは,universal arrow $\opair{r \in \cat C}{\nu \colon \Delta r \nattr F}$ from $\Delta$ to $F$ のことである.
  \[
    r = \iLim F = \lim F
  \]
  などと書く.$\nu$ を\defterm{limiting cone}, of \defterm{universal cone} to $F$ という.
  \begin{comdia}
    F_i \arrow[dd, leftarrow, "F u"'] \arrow[rrd, leftarrow, "\nu_i"] \arrow[rrrrd, bend left=20, leftarrow, "\tau_i"] \\
    & & \iLim F \arrow[rr, dashleftarrow, "\uniq t"] & & c \\
    F_j \arrow[rru, leftarrow, "\nu_j"'] \arrow[rrrru, bend right=20, leftarrow, "\tau_j"']
  \end{comdia}
\end{define}

次のようなnatural isomorphisms in $c$ がある:
\begin{align*}
  \cat C (c, \iLim F) \iso & \Nat (\Delta c, F) = \Cone (c, F), \\
  \Cone (F, c) = & \Nat (F, \Delta c) \iso \cat C (\dLim F, c).
\end{align*}

\subsubsection*{Products}

$J = \{ 0, 1 \}$ をdiscrete categoryとし,$F \colon \{ 0, 1 \} \to \cat C$ をfunctorとする.$F$ は,objects $\opair a b$ of $\cat C$ と見なせる.The limit of $F = \opair a b$ を,\defterm{product of $a$ and $b$}といい,
\[
  \iLim F = a \times b
\]
と書く.

$a \times b$ は,\defterm{projections}と呼ばれる二つの普遍射 $p \colon a \times b \to a$ と $q \colon a \times b \to b$ を持つ:
\begin{comdia}
  a \arrow[rr, leftarrow, "p"] \arrow[rrdd, leftarrow, "f"'] & & a \times b \arrow[rr, "q"] \arrow[dd, dashleftarrow, "\uniq h"] & & b \\
  \\
  & & c \arrow[rruu, "g"']
\end{comdia}

次の全単射がある:
\[
  \cat C (c, a) \times \cat C (c, b) \iso \cat C (c, a \times b), \quad \opair f g \mapsto h,
\]
with inverse $h \mapsto \opair{p h}{q h}$. $f, g$ を $h$ の\defterm{components}という.

\begin{exs}
\item $\cat C = \Set, \Tops, \Cat, \Grp, \Rng, \lMod$, \etcetra, では,productsはdirect products(に適切な構造を入れたもの)である.
\end{exs}

\subsubsection*{Infinite Products}

$J = X$ をsetとすると,functor $F \colon X \to \cat C$ は,単に $X$-indexed family $\fmly{a_x \in \cat C}_{x \in X}$ である.また,cone $\tau \colon \Delta c \to F$ to $F$ from $c$ とは,$X$-indexed family $\fmly{\tau_x \colon c \to a_x}_{x \in X}$ of arrowsである.

\begin{define}
  $\cat C$ をcategory,$X$ をsetとする.$F = \fmly{a_x}_{x \in X} \colon X \to \cat C$ をfunctorとする.The limit $\opair{\iLim F}{p}$ of $F$ を\defterm{product of $a_x$}といい,$p = \fmly{p_x \colon \iLim F \to a_x}_{x \in X}$ を\defterm{projections}という.
  \[
    \iLim F = \prod_{x \in X} a_x = \prod a_x = \prod a
  \]
  などと書く.
  \begin{comdia}
    a_x \arrow[rr, leftarrow, "p_x"] \arrow[rrdd, leftarrow, "f_x"'] & & \prod_{x' \in X} a_{x'} \arrow[dd, dashleftarrow, "\uniq \ext f"] \\
    \\
                                                                     & & c
  \end{comdia}
\end{define}

次のnatural isomorphisms in $c$ がある:
\[
  \prod_x \cat C (c, a_x) \iso \cat C \left( c, \prod_x a_x \right), \quad \fmly{f_x}_x \mapsto \ext f,
\]
with inverse $\ext f \mapsto \fmly{p_x \ext f}_x$. $f_x$ を $\ext f$ の\defterm{components}という.

\subsubsection*{Powers}

$b = \fmly{b_x = a}_{x \in X} \in \cat C^X$ のproductを\defterm{power of $a$}といい,$a^X \defeq \prod_x a$ と書く.このとき,
\[
  \cat C (c, a)^X \iso \cat C (c, a^X), \quad \text{naturally in}\ c.
\]

\subsubsection*{Equalizers}

$J = \downdownarrows$ とすると,functor $F \colon \downdownarrows \to \cat C$ は,射 $f, g \colon b  \rightrightarrows a$ in $\cat C$ である.

\begin{define}
  $\cat C$ をcategoryとし,$F = \opair f g \colon \downdownarrows \to \cat C$ をfunctorとする.The limit $\opair{e = \iLim F}{u \colon e \to b}$ of $F$ を\defterm{equalizer of $F$}という.
  \begin{comdia}
    e \arrow[rr, "u"] \arrow[dd, dashleftarrow, "\uniq \ext h"'] & & b \arrow[rr, shift left, "f"] \arrow[rr, shift right, "g"'] & & a \\
    \\
    c \arrow[rruu, "h"]
  \end{comdia}
\end{define}

\begin{exs}
\item $\cat C = \Set, \Tops, \Grp, \Ab, \Rng, \lMod$, \etcetra, とする.$f, g \colon b \rightrightarrows a$ のequalizerとは,$e = \{ x \in b \deli f x = g x \} \inc b $ である.特に,$\Grp, \Ab, \Rng, \lMod$, \etcetra, では,$\ker (f - g) \inc b$ である.
\end{exs}

\subsubsection*{Pullbacks}

$J = \pushoutc$ とすると,functor $F \colon \pushoutc \to \cat C$ は,射 $\opair f g \colon \opair b d \to \opair a a$ のことである.$F$ のconeは,
\begin{comdia}
  c \arrow[rr, "k"] \arrow[dd, "h"'] & & d \arrow[dd, "g"] \\
  \\
  b \arrow[rr, "f"'] & & a
\end{comdia}
である.

\begin{define}
  $\cat C$ をcategory,$F = \opair f g \colon \pushoutc \to \cat C$ をfunctorとする.The limit $\opair{r = \iLim F}{\opair p q \colon \opair c c \to \opair b d}$ of $F$ を\defterm{pullback of $f, g$}といい,
  \[
    r = b \times_a d
  \]
  などと書く.
  \begin{comdia}
    & & b \arrow[rrdd, leftarrow, "p"] \arrow[rrrrdd, bend left=20, leftarrow, "h"] \\
    \\
    a \arrow[rruu, leftarrow, "f"] \arrow[rrdd, leftarrow, "g"'] & & & & b \times_a d \arrow[rr, dashleftarrow, "\uniq t"] & & s \\
    \\
                                                                 & & d \arrow[rruu, leftarrow, "q"'] \arrow[rrrruu, bend right=20, leftarrow, "k"']
  \end{comdia}
\end{define}

\begin{define}
  $\cat C$ をcategory,$f \colon b \to a$ とする.The \defterm{kernel pair of $f$}とは,pullback of $\opair f f$ のことである.
  \begin{comdia}
    r \arrow[rr, shift left, "u"] \arrow[rr, shift right, "v"'] \arrow[dd, dashleftarrow, "\uniq t"] & & b \arrow[rr, "f"] & & a \\
    \\
    s \arrow[rruu, shift left, "h"] \arrow[rruu, shift right, "k"']
  \end{comdia}
\end{define}

\subsubsection*{Others}

$J = 0$ をempty categoryとすると,functor $F \colon 0 \to \cat C$ は唯一つしかなく(empty functor),coneは単にobject $c \in \cat C$ である.従って,limit of the empty functorは,terminal object $t \in \cat C$ のことである.

Limitsを,functor $J \to \cat C$ に対してではなく,diagram $D \colon G \to U \cat C$, where $G$ is a graph and $U \colon \Cat \to \Grph$ is a forgetful functor, に対して考えた方が便利な場合もある.このとき,cone $\mu \colon c \nattr D$ とは,各 $i \in G$ に対して射 $\mu_i \colon c \to D_i$ が定まり,任意の $h \colon i \to j$ に対して $D h \circ \mu_i = \mu_j$ となるものである.
\begin{comdia}
  D_j \arrow[dd, leftarrow, "F h"'] \arrow[rrd, leftarrow, "\lambda_j"] \arrow[rrrrd, bend left=20, leftarrow, "\mu_j"] \\
  & & \iLim D \arrow[rr, dashleftarrow, "\uniq t"] & & c \\
  D_i \arrow[rru, leftarrow, "\lambda_i"'] \arrow[rrrru, bend right=20, leftarrow, "\mu_i"']
\end{comdia}

\subsubsection*{Exercises}


\printindex
\end{document}
